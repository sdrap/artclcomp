%%%%%%% This is a template file for the use of artcomp.cls

%%%%%%
%%	The underlying environment of artclcomp is scrcartcl 
%%	from the bundle Koma-script
%%	therefore the same options can be passed to artclcomp

\documentclass[DIV=classic,a4paper]{artclcomp}

%%%%%%
%%	Personal head with minimum of usefull packages,
%%	ntheorem environment and shortenings

%% Load Packages


\usepackage{times}
\usepackage{eurosym}
\usepackage{textcomp}
\usepackage{mathrsfs}
\usepackage{graphicx}
\usepackage{enumitem}
\usepackage{stmaryrd}
\usepackage{epsfig}
\usepackage[usenames,dvipsnames]{xcolor}


%%Corrections
%\usepackage[normalem]{ulem}
%\newcommand{\str}[1]{\textcolor{red}{\sout{#1}}}


% MATH -----------------------------------------------------------
\newcommand{\norm}[1]{\left\Vert#1\right\Vert}
\newcommand{\abs}[1]{\left\vert#1\right\vert}
\newcommand{\Set}[1]{\ensuremath{ \left\{ #1 \right\} }}
\newcommand{\set}[1]{\ensuremath{ \{ #1 \} }}
\newcommand{\R}{\mathbb{R}}
\newcommand{\N}{\mathbb{N}}
\renewcommand{\mid}{\,|\,}
\newcommand{\Mid}{\:\big | \:}
\newcommand{\mmid}{\;\Big | \;}
\newcommand{\MMid}{\;\bigg | \;}

\newcommand*{\cadlag}{c\`adl\`ag}
\newcommand*{\caglad}{c\`agl\`ad}
\newcommand*{\ladlag}{l\`adl\`ag}

\newcommand*{\dom}{{\rm dom}}
\newcommand*{\epi}{{\rm epi}}
\newcommand*{\hypo}{{\rm hypo}}
\newcommand*{\gr}{{\rm gr}}
\newcommand*{\cl}{{\rm cl}}
\newcommand*{\co}{{\rm co}}

\DeclareMathOperator*{\esssup}{ess\,sup}
\DeclareMathOperator*{\essinf}{ess\,inf}
\DeclareMathOperator*{\esslimsup}{ess\,limsup}
\DeclareMathOperator*{\essliminf}{ess\,liminf}
\DeclareMathOperator*{\argmin}{arg\,min}
\DeclareMathOperator*{\argmax}{arg\,max}
\DeclareMathOperator*{\simequiv}{\sim}


\newcommand{\C}{\mathbb{C}}


\usepackage[hyperref,amsmath,thmmarks]{ntheorem}

% THEOREMS -------------------------------------------------------
%\renewtheorem{Proof}{proof}
\theoremseparator{.}
\newtheorem{theorem}{Theorem}[section]
\newtheorem{corollary}[theorem]{Corollary}
\newtheorem{conjecture}[theorem]{Conjecture}
\newtheorem{assumption}[theorem]{Assumption}
\newtheorem{lemma}[theorem]{Lemma}
\newtheorem{proposition}[theorem]{Proposition}

\theorembodyfont{\upshape}
\newtheorem{definition}[theorem]{Definition}

\theoremsymbol{\ensuremath{\lozenge}}
\newtheorem{example}[theorem]{Example}

\theoremsymbol{\ensuremath{\blacklozenge}}
\theoremheaderfont{\itshape}
\newtheorem{remark}[theorem]{Remark}

\theoremsymbol{\ensuremath{\blacklozenge}}
\theoremheaderfont{\itshape}
\newtheorem{remarks}[theorem]{Remarks}

\theoremsymbol{\ensuremath{\square}}
\theoremheaderfont{\itshape}
\theoremstyle{nonumberplain}
\newtheorem{proof}{Proof}


\numberwithin{equation}{section}




%%%%%%
%%	Package to fill the abstract with Latin non-sense.

\usepackage{lipsum}



\begin{document}

%%%%%%  Title section
%% Several footnotes to the title can be provided.
%% the key is \tnoteref{somekey}
%% after the title provide \tnotetext[somekey]{Some text as footnote}
%% if no key then no footnote.

\title{The Ultimate Approach to a New Concept.\tnoteref{t}}
\tnotetext[t]{This paper was presented in blablablabla}


%%%%%% Author section
%%	Way to do it is \author[key1,key2,key3]{Name Firstname}
%% where keys are corresponding to one of the optional entry
%% address
%% eMail
%% myThanks
%% The same key can be used for several authors (for instance for the common thanks or common address
%% in which case it will be print once but reference to each of the corresponding author.

\author[a,1,s]{Rolf Guntermaier}
\author[a,2,t]{Giovani da Pintoresque}
\author[b,3,s]{Marcel Duchamp}

\address[a]{Great West University, far End street 6, 10064 Leduc, Europe}
\address[b]{Middle North University, A3D4 street 8, 84353 Montrant-Ville, USA}

\eMail[1]{rgunt@gw-university.edu.eu}
\eMail[2]{gio@gw-university.edu.eu}
\eMail[3]{MarcelDuchamp@mn-university.edu}

\myThanks[s]{Funding: Great Bank of Tangtui}
\myThanks[t]{Funding: Research Project OTGIRTNSA, Project X645DR-88}


%%%%%%% Abstract Section
%% Abstract as usual
%% \keyWords as list of keywords (careful the W in keyWords is capitalised.
%%

\abstract{
	 \lipsum[1]
}
\keyWords{Something Interesting, Risk Measures, Profoundly Irrelevant}


%%%%%%%% Paper Infos (optional if not given they won't appear.
%%	\ArXiV give the number of the arxiv reference, it provides the link
%% \keyAMSClassification list of AMS Classification
%% \keyJELClassification list of JELClassification
%%	



\ArXiV{1212.6732}
\keyAMSClassification{83H54, 73D12}
\keyJELClassification{5344, 6355}
\maketitle

\section{Introduction}
\lipsum[2-5]

\section{Some Math}
Let $M, N:\tilde{\Omega}\times [0,T]\to \R$ be $(\tilde{\mathcal{F}}_t)$-adapted processes.
The process $M$ is called \cadlag, \caglad~or \ladlag~if the paths of $M$ are  \cadlag, \caglad~or \ladlag~quasi-surely, respectively.
Given a \ladlag~process, we denote by $M^-$ and $M^+$ its \caglad~and \cadlag~version respectively, that is
\begin{gather*}
	M^-_t:=\lim_{s\nearrow t}M_s,\quad \text{for }t \in ]0,T],\quad\text{and}\quad M^-_0:=M_0,\\
	M^+_t:=\lim_{s\searrow t}M_s,\quad \text{for }t \in [0,T[,\quad\text{and}\quad M^+_T:=M_T,
\end{gather*}
outside the polar set where $M$ is not \ladlag.
Two $(\tilde{\mathcal{F}}_t)$-adapted processes $M,N:\tilde{\Omega}\times [0,T]\to \R$ are modifications of each others, if $M_t=N_t$, for all $t \in [0,T]$.
We say that $M$ is a \emph{supermartingale} or a \emph{strong supermartingale}, if $M(\theta)$ is a supermartingale or a strong supermartingale, for all $\theta \in \Theta$, respectively.
See \cite[Appendix I]{Dellacherie1982} for a definition of strong supermartingales.

Let us define the following sets of value and control processes:
\begin{itemize}
	\item $\mathcal{S}$ is the set of $(\mathcal{F}_t)$-adapted \ladlag~processes $Y:\Omega \times [0,T]\to \R$;
	\item $\tilde{\mathcal{S}}$ is the set of \ladlag~processes $Y:\tilde{\Omega}\times [0,T]\to \R$, which have a modification $\hat Y$ satisfying $\hat Y_t \in C(\tilde{\mathcal{F}}_t)\cap L^1_b(\tilde{\mathcal{F}}_t)$, for all $t \in [0,T]$, and such that $Y(\theta)$ is optional, for all $\theta \in\Theta$.
	\item For any $\theta \in \Theta$, let  $\mathcal{L}(\theta)$ be the set of $(\mathcal{F}_t)$-predictable processes $Z:\Omega \times [0,T]\to \R^d$ such that $P\left[\int_{0}^{T}\|Z_u \theta_u^{1/2}\|^2 du<\infty\right]=1$.
	\item $\tilde{\mathcal{L}}$ is the set of $(\tilde{\mathcal{F}}_t)$-predictable processes $Z:\tilde{\Omega}\times [0,T]\to \R^d$ such that $Z(\theta)\in \mathcal{L}(\theta)$, for all $\theta \in \Theta$.
\end{itemize}
%A \emph{generator} is a jointly measurable function g from $\tilde \Omega \times [0, T ]\times \mathbb{R}\times \mathbb{R}^{1\times d}$ to $\mathbb{R}\cup  \{+\infty\}$ where $\tilde \Omega \times [0, T ]$ is endowed with the progressive $\sigma$-field.
A \emph{generator} is a jointly measurable function g from $\tilde \Omega \times [0, T ]\times \mathbb{R}\times \mathbb{R}^{1\times d}$ to $\mathbb{R}\cup  \{+\infty\}$ such that the mapping $(s,\omega, \theta)\mapsto g_s(\omega,\theta,y,z):([0,t]\times \tilde \Omega,\mathcal{B}([0,t])\otimes \tilde{\mathcal{F}}_t)\rightarrow (\mathbb{R}^d,\mathcal{B}(\mathbb{R}^d))$ is measurable, for each $t$, for all $(y,z)\in\mathbb{R}^{d+1}$.
We say that a generator $g$ is
\begin{enumerate}[label=\textsc{(Pos)},leftmargin=40pt]
	\item \label{cond00} positive, if $g\left(\theta, y,z \right)\geq 0$;
\end{enumerate}
\begin{enumerate}[label=\textsc{(Lsc)},leftmargin=40pt]
	\item \label{condlsc} if $(y,z)\mapsto g(\theta,y,z)$ is lower semicontinuous;
\end{enumerate}
\begin{enumerate}[label=\textsc{(Mon)},leftmargin=40pt]
	\item \label{cond03} increasing, if $g\left(\theta,y,z  \right)\geq g\left(\theta, y^\prime ,z\right)$, whenever $y\geq y^\prime$; 
\end{enumerate}
\begin{enumerate}[label=\textsc{(Mon${}^\prime$)},leftmargin=40pt]
	\item \label{cond02} decreasing, if $g\left(\theta,y,z  \right)\leq g\left( \theta, y^\prime ,z\right)$, whenever $y\geq y^\prime$; 
\end{enumerate}
\begin{enumerate}[label=\textsc{(Con)},leftmargin=40pt]
	\item \label{cond01} convex, if $g\left(\theta,y,\lambda z+\left( 1-\lambda \right)z^\prime \right)\leq \lambda g\left( \theta, y,z \right)+\left( 1-\lambda \right)g\left( \theta, y,z^\prime \right)$, for all $\lambda \in \left( 0,1 \right)$;
\end{enumerate}
\begin{enumerate}[label=\textsc{(Nor)},leftmargin=40pt]
	\item \label{cond04} normalized, if $g\left(\theta,y,0 \right)=0$;
\end{enumerate}
$P\otimes dt$-almost surely, for all $y,y'\in\R$, all $z,z^\prime  \in \R^{1 \times d}$ and all $\theta \in \Theta$.

A pair $(Y,Z)\in\tilde{{\cal S}}\times \tilde{\mathcal{L}}$ is said to be a \emph{supersolution} of the BSDE with generator $g$ and terminal condition $\xi\in L^0(\tilde{\cal F}_T)$, if
\begin{equation}\label{eq:central:ineq:rob}
	Y_\sigma(\theta)-\int_\sigma^\tau g_{u}(\theta,Y_u(\theta),Z_u(\theta))du+\int_\sigma^\tau Z_{u}(\theta) d\tilde{W}_{u}(\theta) \geq Y_\tau(\theta)\quad\text{and} \quad Y_T(\theta)\geq\xi(\theta),
\end{equation}
for all $\sigma,\tau \in \mathcal{T}$, with $\sigma\leq \tau$, and for all $\theta \in \Theta$.
For such a supersolution $(Y,Z)$, we call $Y$ the \emph{value process} and $Z$ its \emph{control process}.
In order to exclude doubling strategies we only consider control processes, which are admissible, that is $\int Z(\theta)d\tilde W(\theta)$ is a supermartingale, for all $\theta\in\Theta$.
We denote the set of such supersolutions by
\begin{equation}\label{set:rob:A:B}
	\mathcal{A}(\xi,g)= \set{ (Y,Z) \in\tilde{\mathcal{S}}\times \tilde{\mathcal{L}} : Z \text{ is admissible and } \eqref{eq:central:ineq:rob} \text{ holds}}.
\end{equation}
A pair $(Y,Z)$ is said to be a \emph{minimal supersolution}, if $(Y,Z)\in \mathcal{A}(\xi,g)$, and if for any other supersolution $(Y',Z')\in\mathcal{A}(\xi,g)$, holds $Y_t \leq Y^{\prime}_t$, for all $t\in[0,T]$. 
The natural candidate for the value process of a minimal supersolution is the infimum, that is
\begin{equation}
	\hat{\mathcal{E}}_t(\xi):=\hat{\mathcal{E}}^{g}_t(\xi):=\inf \set{Y_t : \left( Y,Z \right)\in \mathcal{A}(\xi,g)}, \quad t \in \left[ 0,T \right].
	\label{eq:defi:phi:rob1:B}
\end{equation}
The goal is to find a modification $\mathcal{E}(\xi)$ of $\hat{\mathcal{E}}(\xi)$ that belongs to $\tilde{\mathcal{S}}$ and some admissible process $Z$ in $\tilde{\mathcal{L}}$ such that $(\mathcal{E}(\xi), Z)$ fulfills \eqref{eq:central:ineq:rob}, that is $(\mathcal{E}(\xi),Z)$ is a minimal supersolution.


\section{Conclusion}
\lipsum[5-8]


%% I you prefer (recommended) an input organisation of your article
%\input{introduction}
%\input{chap01}
%\input{chap02}
%\input{chap03}
%\input{chap04}
%\input{chap05}

%% The bibliography is as usual.
%% Just choose the style that you like, it should be natbib compatible.
\citet{foellmer02}\citet{artzner01}\citep{Dellacherie1978,DHK1101}
\bibliographystyle{econometrica}
%\bibliographystyle{abbrvnat}
%%\bibliographystyle{./bibfiles/plaindin}
%%\bibliographystyle{./bibfiles/unsrtdin}
\bibliography{bib}

\end{document} 
